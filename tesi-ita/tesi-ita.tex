\documentclass[12pt,a4paper,twoside]{book}

\usepackage[a4paper,inner=3.5cm,outer=2.5cm]{geometry}

\begin{document}
\chapter{Sommario del Progetto}

Questa tesi presenta uno studio delle applicazioni della crittografia omomorfica nei protocolli di comunicazione orientati alla privacy. Durante l'analisi, ci siamo soffermati sull'esaminare e confrontare le differenze di prestazioni tra questi protocolli, considerando un possibile utilizzo in un contesto reale. Ci siamo quindi concentrati sull'identificazione di criticità e punti deboli nei modelli \textit{Publisher-Subscriber} e \textit{Request-Response}.

Una parte centrale del lavoro è stata il confronto tra i diversi modelli di "encoding" dei dati geografici, al fine di renderne compatibile l'utilizzo con la crittografia omomorfica. Non è stato possibile adottare direttamente un modello basato su dati spaziali GPS, in quanto questi sono rappresentati tramite numeri a virgola mobile, non supportati nativamente dalla crittografia omomorfica. Per questo motivo, è stato necessario adottare un modello alternativo, basato su un encoding di numberi interi: più precisamente, un modello a griglia che enumerasse le celle tramite "space-filling curves".

Entrando nello specifico, è stato ideato e progettato un protocollo di comunicazione che combini entrambe le modalità discusse precedentemente, massimizzando le prestazioni e riducendo lo spreco di risorse (in particolare nella comunicazione). Questo protocollo si ispira al modello Location Aware MQTT, un'estensione di MQTT pensata per l'utilizzo con sensori e dispositivi mobili, i quali pubblicano e ricevono posizioni in tempo reale. La motivazione alla base di questa tesi è stata la necessità di applicare un meccanismo di crittografia a questo tipo di protocolli, al fine di garantire la privacy di tutti gli attori coinvolti.

Durante lo studio sono state affrontate le principali problematiche intrinseche alla crittografia omomorfica: la gestione delle chiavi e delle chiavi di traduzione, l'elevato consumo di risorse computazionali (non sempre disponibili su dispositivi mobili e IoT) e la tutela della sicurezza dei dati. Per ovviare ai primi due problemi, è stato adottato un sistema di gestione delle chiavi semi-centralizzato, combinato con un proxy re-encryptor che alleggerisse il carico computazionale sui sensori e, al contempo, garantisse l'anonimato dei dati. Per quanto riguarda il terzo problema, è stato necessario garantire che i singoli client presenti all'interno del sistema siano gli unici in grado di decifrare i dati. Questo accorgimento ha risolto uno dei problemi più gravi presenti nella versione precedente di HE LA-MQTT, nella quale i dati venivano decifrati dalla Certification Authority, consentendo, a causa di un errore di progettazione, la fuoriuscita di dati sensibili.

\end{document}
