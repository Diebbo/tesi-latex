\documentclass[12pt,a4paper,twoside]{book}


% \usepackage[utf8]{inputenc}
\usepackage[a4paper,inner=3.5cm,outer=2.5cm]{geometry}

\usepackage[titletoc,title,toc,page]{appendix}
\usepackage{tabularx}
\usepackage{array}
\usepackage{amsmath}
\usepackage{unicode-math}

\usepackage{verbatim}
\usepackage{placeins}
\usepackage{listings}
\usepackage{hyperref}
\usepackage[english]{babel}
\usepackage{tikz}
\usepackage{parskip}

\usepackage{graphicx}
\usepackage{blindtext}
\usepackage{chngcntr}
\counterwithin{table}{chapter}

\usepackage{newlfont}
\usepackage{fancyhdr}
\usepackage{indentfirst}
% \usepackage[utf8]{inputenc}
\usepackage{float}
\usepackage{hyperref}
\usepackage[capitalize,noabbrev]{cleveref}
\usepackage{soul}
\usepackage[font=footnotesize,labelfont=bf]{caption}

\usepackage{multirow}
\usepackage{hyphenat}
\hyphenation{mate-mati-ca recu-perare}

\usepackage{lscape} 

\usepackage{natbib}
\bibliographystyle{alpha}
\setcitestyle{super,open={[},close={]}}

\newcommand{\rom}[1]{\uppercase\expandafter{\romannumeral #1\relax}}

\usepackage{pdfpages}

\begin{document}
% Per spostare i vari elementi più su o più giù gioca con i valori di vspace che ci sono tra uno e l'altro
\pagestyle{empty}
\newgeometry{
    left=20mm,
    right=20mm,
    top=20mm,
    bottom=20mm
}

\begin{titlepage}

\begin{center}

% marchio di ateneo
\includegraphics[width=6.5cm,height=4.7cm]{img/marchio-di-ateneo.png}

\vspace{10mm}

% \large is 12pt
{\large{\bf{Dipartimento di Scienze ed Ingegneria Informatica}}}

\vspace{5mm}

% \Large is 14.4pt
{\Large{\bf{Corso di Informatica}}}

\vspace{15mm}

{\Huge{\bf A Study on Applications of }}\\
\vspace{3mm}
{\Huge{\bf Homomorphic Encryption in }}\\
\vspace{3mm}
{\Huge{\bf Privacy-Preserving Protocols}} \\

\end{center}

\vspace{10mm}

\begin{minipage}[t]{0.50\textwidth}
{\large{\bf Relatore: \\ Chiar.mo Prof.\\ FEDERICO MONTORI}}

\vspace{3mm}

{\large{\bf Correlatore: \\ Chiar.mo Prof.\\ SAVERIO GIALLORENZO}}
\end{minipage}
\hfill
\begin{minipage}[t]{0.40\textwidth}\raggedleft
{\large{\bf Presentata da: \\ DIEGO BARBIERI}}
\end{minipage}

\vspace{30mm}

\rule[0.5cm]{15.8cm}{0.6mm}

\begin{center}
{\large{\bf Sessione di Luglio 2025 \\}}
{\large{\bf Anno Accademico 2024/2025\\}}
\end{center}

\end{titlepage}

\restoregeometry
\newpage

% https://tex.stackexchange.com/questions/26538/words-scattered-randomly-in-on-coverpage
\begin{tikzpicture}[overlay,remember picture,shift=(current page.center)]
\pgfmathsetseed{3}


\foreach [count=\count] \word in {Homomorphic Encryption, Privacy, Network Protocols, Zero-Trust, Location} {
\node [
    xshift={(mod(\count,3)-1)*(\paperwidth/4)},
    yshift={(mod(\count,7)-3)*(\paperwidth/6)},
    xshift=rand*4cm,
    yshift=rand*2cm,
    % rotate=rand*35,
    % opacity=rnd*0.5+0.125,
    font=\large] {\word};
}
\end{tikzpicture}
\newpage

\topmargin=6.5cm
\begin{flushright}
\emph{
\LARGE{A me, al mio culone }\\\vspace{2mm}
\LARGE{e al mio di-ego}\\\vspace{3mm} 
}
\end{flushright}
\newpage~\newpage
\pagenumbering{gobble}


% \newgeometry{
%     left=20mm,
%     right=20mm,
%     top=20mm,
%     bottom=20mm
% }


\chapter*{Abstract}
This thesis presents a comprehensive study on the application of homomorphic encryption (HE) in privacy-preserving communication protocols, with a particular focus on location-based services. We investigate and compare various network protocols that leverage HE to enhance user privacy, analyzing their effectiveness and identifying the practical limitations they face in real-world scenarios. 
A central part of this work explores different encoding strategies that enable location data to be securely processed under homomorphic encryption schemes. We examine how these strategies impact both the privacy guarantees and the computational overhead of the protocols. Special attention is given to the integration of HE in publish/subscribe and request/response paradigms, and to the trade-offs that would arise.

As a case study, we design and implement a protocol that allows mobile clients to discover nearby parking spots without revealing their precise location to the server. The protocol was designed starting from the previous analysis of existing solutions, such as LA-MQTT, and incorporates homomorphic encryption to ensure that the server can process location queries without accessing sensitive data. We evaluate the performance of our protocol in terms of computational efficiency and communication overhead, comparing it with existing solutions.

\topmargin=-1cm
\tableofcontents
\thispagestyle{empty}
\listoftables
\thispagestyle{empty}
\listoffigures
\thispagestyle{empty}
\newpage~\newpage

\pagenumbering{arabic}
% \setcounter{chapter}{1}
\raggedbottom
\chapter{Introduction} \label{chap:intro}

% Structure of the introduction:
% - What is the problem?
%     - Hide user position in LA-MQTT
%     - maximize privacy over performance
% - Why is it important?
%     - Privacy is a fundamental right in location-based services
%     - Zero-trust scenarios are becoming more common
% - How do i solve it?
%     - Use homomorphic encryption to protect user position
%     - Combine publish/subscribe with request/response protocols
%
% narrow down the scope.
%
% General > Specific > More Specific

In recent years, location-based services have become increasingly prevalent in our daily lives. From finding nearby accommodations or points of interest to accessing services tailored to our geographical context, location-awareness is playing an essential role in modern applications. However, this convenience comes at a cost: the necessity of sharing users' location data with third parties, often leading to significant privacy concerns.

Exposing our location could lead to unwanted tracking, profiling, and even physical harm. Therefore, it is crucial to adopt technologies that can preserve user privacy even in adversarial settings. For this purpose, homomorphic encryption (HE) emerges as a promising solution that allows computations to be performed on encrypted data without revealing the underlying plain text. This means that sensitive information, such as our location, can be processed without exposing it to the server. This type of encryption is becoming extremely popular in fields such as cloud computing, where data is often stored and processed by third-party providers. 
At the same time, we are witnessing a shift towards zero-trust protocols, where no middle man is trusted and there is no central authority that can be relied upon to handle sensitive data. 


The focus of this thesis regards the use of homomorphic encryption in privacy-preserving protocols, considered in the context of location-based services. In particular, I have developed a protocol that allows clients to find nearby parking spots without revealing their position to the server. The protocol is based on the use of homomorphic encryption to protect the client position and the parking spots, and it uses a combination of publish/subscribe and request/response protocols to allow clients to find nearby parking spots without revealing their position to the server.

The motivation for creating a privacy-preserving protocol comes from the necessity of implementing inside the Location Aware MQTT (LA-MQTT) protocol~\cite{montori2022lamqtt} this type of mechanism. LA-MQTT is an extension of MQTT, optimized to work with geofence data and Location Data Sources. The protocol was originally meant to be used with other types of privacy standards that would only obfuscate the client position, but not hide it completely.

\pagestyle{plain}
\setcounter{page}{1}


\chapter{Background}
\section{Request-Response Protocol}
The Request-Response protocol is a fundamental communication pattern in distributed systems where a client sends a request to a server and waits for a response. This synchronous communication model is characterized by its simplicity and direct interaction between parties, making it suitable for operations requiring immediate feedback and confirmation.

\subsection{HTTP}
HTTP (Hypertext Transfer Protocol) is the foundation of data communication on the World Wide Web. It operates as a request-response protocol, allowing clients to request resources from servers and receive responses. HTTP supports various methods (GET, POST, PUT, DELETE) for different types of operations and is extensible through headers and status codes.

\subsubsection{HTTPS}
HTTPS is a version of HTTP built on the SSL/TLS protocol, providing secure communication over a computer network. It encrypts data exchanged between clients and servers, ensuring confidentiality and integrity. HTTPS is essential for protecting sensitive information from eavesdropping and tampering.

One of the key features of HTTPS is its use of certificates to authenticate the server, ensuring that clients are communicating with the intended entity. This is possible through a relatively high usage of computational resources, which is a trade-off for the enhanced security it provides.

\subsection{Request-Response in IoT devices}
HTTPS is a commonly used choice also for IoT devices. However, its usability is often limited due to a number of factors:
\begin{itemize}
    \item \textbf{Resource Constraints\cite{mazhar2023iotsecurity}}: Encrypting and decrypting certificates standards (RSA, EEC, AES) can be computationally expensive, which is a significant concern for IoT devices with limited processing power and memory.
    \item \textbf{Lack of Secure Firmware Updates\cite{cyberark2024iot}}: Many IoT devices do not support secure firmware updates, making it difficult to patch vulnerabilities in the HTTPS implementation.
    \item \textbf{Weak or Nonexistent Certificate Validation\cite{bishopfox2020weakcertificates}}: Many IoT devices do not validate server certificates properly, leading to potential vulnerabilities.
\end{itemize}

\section{Publish-Subscribe Protocol}
The Publish-Subscribe (Pub/Sub) protocol is an asynchronous messaging pattern where senders (publishers) categorize messages into topics without knowledge of the receivers (subscribers). Subscribers express interest in specific topics and receive messages published to those topics. This decoupled architecture enables scalable and flexible communication in distributed systems.

\subsection{MQTT}
MQTT (Message Queuing Telemetry Transport) is a lightweight, open-source messaging protocol designed for constrained devices and low-bandwidth, high-latency networks. It implements the publish-subscribe pattern over TCP/IP, providing three quality of service levels for message delivery and supporting various security features.

The protocol defines three main network entities:
\begin{itemize}
    \item \textbf{Message Broker}: The central component that manages message routing between publishers and subscribers. It receives messages from publishers and forwards them to subscribers based on their subscriptions.
    \item \textbf{Publisher}: A client that sends messages to the broker on specific topics. 
    \item \textbf{Subscriber}: A client that expresses interest in specific topics and receives messages published to those topics by the broker.
\end{itemize}

\subsubsection{MQTT Quality of Service Levels}

MQTT provides three quality of service (QoS) levels to ensure message delivery reliability:
\begin{itemize}
    \item \textbf{QoS 0 (At most once)}: The message is delivered at most once, with no acknowledgment from the receiver. This level is suitable for applications where occasional message loss is acceptable.
    \item \textbf{QoS 1 (At least once)}: The message is guaranteed to be delivered at least once, with acknowledgment from the receiver. This level ensures that messages are not lost but may result in duplicates.
    \item \textbf{QoS 2 (Exactly once)}: The message is guaranteed to be delivered exactly once, using a four-step handshake process. This level provides the highest reliability but incurs more overhead.
\end{itemize}

\subsubsection{MQTT Security}

As we mentioned, one of the key features of MQTT is the possibility to scale the protocol to fit the needs of the application. This is possible by using different security features, such as TLS/SSL for secure communication, authentication mechanisms to verify client identities, and access control lists to restrict topic access. These features help protect against unauthorized access and ensure the integrity of messages exchanged between clients.

\subsubsection{MQTT Workflow}

\begin{figure}[h]
    \centering
    \includegraphics[width=10.5cm,height=4.7cm]{img/mqtt-workflow-example.png}
    \caption{MQTT Workflow}
    \label{fig:mqtt-workflow}
\end{figure}

Generally, the MQTT workflow starts with the client establishing a connection to the broker, using a TCP/IP connection with optional TLS/SSL security. Once connected, the client can publish messages to specific topics or subscribe to topics of interest. The broker then routes messages to subscribers based on their subscriptions.

\subsection{LA-MQTT} \label{sec:la-mqtt}
LA-MQTT (Location-Aware MQTT) extends the standard MQTT protocol by incorporating location-based features. It enables spatial queries and location-aware message routing, making it particularly suitable for IoT applications requiring geographical context in message distribution.

The protocol is first introduced with the purpose of resolving the limitations of traditional MQTT in handling location-based data\cite{montori2022lamqtt}.

\begin{table}[h]
\small
\begin{tabularx}{\linewidth}{|l|X|X|X|p{4cm}|}
\hline
\textbf{API} & \textbf{Subject} & \textbf{MQTT OP} & \textbf{Topic} & \textbf{Payload} \\ \hline
Position publish & MC & Publish & GPS\_DATA & $\{position: ( P_i )\}$ \\ \hline
Topic subscription & MC & Subscribe & $C(i, t_s)$ & * \\ \cline{2-5}
 & MC & Publish & MC\_SUB & $\{ mc: i, topic: t_s \}$ \\ \hline
Geo fence publish & LDS & Publish & GEO\_FENCE\_DATA & $\{topic: t_s, content: c_s, $\newline$region: g_s, event: e_s\}$ \\ \hline
Content publish & Backend & Publish & $C(i, t_s)$ & $\{content: c_s\}$ \\ \hline
\end{tabularx}
\caption{The LA-MQTT Publish-subscribe Operations}
\label{table:la-mqtt}
\end{table}

The table \ref{table:la-mqtt} summarizes the main operations of the LA-MQTT protocol, highlighting the interactions between clients (MC), location data sources (LDS), and the backend system.

Those operations include:
\begin{itemize}
    \item \textbf{Position Publish}: MC publish their GPS data to the broker, allowing other clients to receive updates on their positions.
    \item \textbf{Topic Subscription}: MCs subscribe to specific topics, enabling them to receive messages related to their areas of interest.
    \item \textbf{Geo fence Publish}: LDSs publish geo fence data, which includes the topic, content, region, and event associated with the Geo fence.
    \item \textbf{Content Publish}: The backend publishes content related to the subscribed topics, forwarding it to the subscribed MCs.
\end{itemize}

LA-MQTT integrates two privacy-preserving strategies within its client-side architecture:
The first strategy is based on randomized location perturbation. This method applies controlled noise to the GPS coordinates before transmission. Specifically, for a given GPS value $P_i$, a user-defined number of decimal digits dd is preserved, while the remaining digits are randomly replaced. This approach balances between:
\begin{itemize}
    \item Privacy Preservation (PP): Higher randomness enhances anonymity.
    \item Spatial Precision (SP): Excessive perturbation can degrade the accuracy of spatial notifications.
\end{itemize}
The second strategy involves the use of dummy updates. Here, the MC alternates between sending real and synthetic (dummy) location data. In each sequence of $b+1$ updates, only one is the actual position; the others are randomly generated or trajectory-based decoys.

\section{Universal Location Referencing}
Universal Location Referencing provides standardized methods for encoding and representing geographical locations. These systems ensure consistent and unambiguous location representation across different applications and platforms.

\subsection{Background: Distance Measures Between Vectors}
\label{sec:background-distances}

In many applications, especially related to positioning and spatial data, it is essential to measure the similarity or dissimilarity between vectors. This is particularly relevant in fields such as machine learning, computer vision, and geographic information systems (GIS). To analyze similarity or dissimilarity between vectors $\mathbf{x}, \mathbf{y} \in \mathbb{R}^n$, three standard distance metrics are:

\begin{description}
  \item[Cosine similarity:]
  \[
    \mathrm{CS}(\mathbf{x}, \mathbf{y}) 
    = \frac{\sum_i x_i y_i}{\|\mathbf{x}\| \cdot \|\mathbf{y}\|}.
  \]
  This measures the angle between vectors and is scale‑invariant.

  \item[Euclidean distance:]
  \[
    \mathrm{ED}(\mathbf{x}, \mathbf{y}) 
    = \sqrt{\sum_i (x_i - y_i)^2}.
  \]
  It represents the straight-line distance but involves a non-linear square-root.

  \item[Manhattan distance:]
  \[
    \mathrm{MD}(\mathbf{x}, \mathbf{y}) 
    = \sum_i |x_i - y_i|.
  \]
  Also known as the $\ell_1$ norm, it sums up absolute differences.
\end{description}
\subsection{Cantor Pairing}

Cantor Pairing is a mathematical technique that uniquely maps two natural numbers to a single natural number (\cref{fig:cantor}) . This bijective function $\pi: \mathbb{N} \to \mathbb{N}$ is particularly useful in computer science for combining two coordinates into a single value while maintaining the ability to recover the original coordinates.

More formally, the Cantor pairing function is defined as:
\[
    \pi(x, y) = \frac{(x + y)(x + y + 1)}{2} + y
\]

Although this function does not preserver algebraic properties, it provides some unique properties derived from the fact that it is segmenting the two-dimensional space into a zig-zag pattern.

\[
    \pi(x, y) + 1 = \pi(x - 1, y + 1)
\]

Moreover, we also need to define the behavior of the function when hits the boundaries of the first quadrant:

\[
    \pi(x, 0) + 1 = \pi(x + 1, 0)
\]

At last we denote the starting point of the Cantor pairing function as \( \pi(0, 0) = 0 \). This means that the function starts at the origin of the two-dimensional space and maps it to zero in the one-dimensional space. The inverse of the Cantor pairing function can be computed as follows:
\[
    \pi^{-1}(z) = \left( \frac{n(n + 1)}{2} - z, z - \frac{n(n + 1)}{2} \right)
\]
Where \( n \) is the largest integer such that \( \frac{n(n + 1)}{2} \leq z \). This allows us to retrieve the original coordinates \( (x, y) \) from the single value \( z \).

This function is widely used in computer science, particularly in data structures and algorithms, where it is necessary to map multi-dimensional data to a single dimension for efficient storage and retrieval. It is also used in various applications such as database indexing, spatial data representation, and cryptography.


\vspace{5mm}

\begin{figure}[h]
    \centering
    \includegraphics[width=6.5cm,height=4.7cm]{img/cantor-pairing.jpg}
    \caption{Visualization of the Cantor pairing function mapping two-dimensional coordinates to a single value}
    \label{fig:cantor}
\end{figure}

\vspace{5mm}

\section{Space Filling Curves}
Space Filling Curves are mathematical curves that pass through every point in a multi-dimensional space. They provide a way to map multi-dimensional data to a single dimension while preserving spatial locality, making them valuable for spatial indexing and data organization.

The most common space-filling curves include (\cref{fig:space-filling}):
\begin{itemize}
    \item \textbf{Z-order Curve}
    \item \textbf{Hilbert Curve}
\end{itemize}

%\vspace{5mm}

\begin{figure}[h]
    \centering
    \includegraphics[width=6.5cm,height=4.7cm]{img/hilbert-z-order.jpg}
    \caption{Comparison of Z-order and Hilbert curves in two-dimensional space}
    \label{fig:space-filling}
\end{figure}

\vspace{5mm}

The Hilbert Curve is particularly notable for its ability to preserve locality, meaning that points that are close in multi-dimensional space remain close in the one-dimensional representation.

On the other hand, the Z-order Curve, preserve the order of positions in a grid-like manner, making it suitable for applications requiring efficient spatial queries. Thus, it's possible to reduce the problem of finding matching points into finding the maximum prefix of two bit strings.

\subsection{Z-order Curve} \label{sec:z-order-curve}
As mentioned, the Z-order curve is one of the most widely used space-filling curves. It maps multi-dimensional data into a single dimension, similar to the Cantor encoding function. Conversely, it has some some useful properties, when applied to spacial data, such as preserving locality and allowing efficient range queries.

\subsubsection{Z-order Encoding}

The encoding process for Z-order involves interleaving the bits of the coordinates of a point in a multi-dimensional space. For example, given a point with coordinates \( (x, y) \), the Z-order encoding can be represented as:
\[
    Z(x, y) = \sum_{i=0}^{n} (x_i \cdot 2^{2i} + y_i \cdot 2^{2i+1})
\]

Where:
\begin{itemize}
    \item \( x_i \) and \( y_i \) are the bits of the binary representation of the coordinates \( x \) and \( y \).
    \item \( n \) is the number of bits used to represent each coordinate.
\end{itemize}

The Z-order curve can also be applied to encode vectors with dimensions greater than two. In this case, the encoding process involves interleaving the bits of all coordinates in a similar manner.


\subsubsection{Z-order Decoding}
The decoding process retrieves the original coordinates from the Z-order encoded value. Given a Z-order value \( Z \), the decoding can be performed by extracting the bits corresponding to each coordinate:
\[
    x = \sum_{i=0}^{n} (Z \pmod{ 2^{2i+2} }) \cdot 2^i
\]
Where:
\begin{itemize}
    \item \( Z \pmod{ 2^{2i+2} } \) extracts the bits corresponding to the \( i \)-th coordinate.
    \item The result is then shifted and combined to reconstruct the original coordinate \( x \).
\end{itemize}

The same process can be used to find the $y$ value by de-interleaving the other bits.

\subsubsection{Z-order Querying}
Let's consider a scenario where we want to find all points within a specific range in a two-dimensional space. We define \( P \) as the set of points in the space, and we want to find all points \( p \in P \) such that:
\[
    p.x \in [x_{min}, x_{max}] \quad \text{and} \quad p.y \in [y_{min}, y_{max}]
\]

We can leverage the Z-order encoding to efficiently query this range. This is possible by calculating the order of the encoding that we want to find. 
By definition we can represent the GPS coordinates of a point \( (x_p, y_p) \) into a point \( e_p \). This point will be represented in the space of \( n \) orders, each contained in \( \mathbb{Z}_4 \).

\subsubsection{Maximum Common Prefix Search}
A key operation in Z-order based spatial queries is finding the maximum common prefix between two Z-order encoded values. This operation is fundamental for determining spatial relationships between points.

Given two Z-order encoded values \( Z_1 \) and \( Z_2 \), we can find the maximum common prefix by following three steps. Firstly, we convert both values to their binary representation. Then, we compare the bits from left to right, stopping at the first position where the bits differ. Finally, we extract the common prefix up to that point. The cited algorithm shows that in the worst case, the maximum common prefix can be found in \( O(n) \) time, where \( n \) is the number of bits in the Z-order encoding.

The length of the common prefix determines the size of the smallest bounding box that contains both points. This property is particularly useful for:
\begin{itemize}
    \item Finding the smallest region containing multiple points
    \item Determining if points are within a certain distance of each other
    \item Optimizing spatial range queries
\end{itemize}

For example, consider two points with Z-order encodings:
\begin{align*}
    Z_1 &= 3310_4 \\
    Z_2 &= 3312_4
\end{align*}
The maximum common prefix is \( 331_4 \), indicating that these points share the same region in the first three orders of the encoded space. We used base 4 because it's the easiest way to visualize which of the four quadrants the point belongs to, as each digit represents a quadrant in a two-dimensional space.

This prefix-based approach can be extended to handle range queries by:
\begin{enumerate}
    \item Encoding the query range boundaries
    \item Finding the maximum common prefix of the range
    \item Generating all possible Z-order values that share this prefix
\end{enumerate}

The efficiency of this approach comes from the fact that we can perform these operations using simple bitwise operations, making it suitable for real-time applications.


\section{Privacy Preserving}
Privacy Preserving techniques ensure the protection of sensitive information while allowing necessary computations and data processing. These methods are crucial in maintaining confidentiality in distributed systems and data analysis. For example, in the context of location-based services, privacy-preserving techniques allow for the sharing of location data without revealing exact coordinates, thus protecting user privacy.

\subsection{Homomorphic Encryption}
Homomorphic Encryption(HE) is a form of encryption that allows specific types of computations to be performed on cipher text, producing an encrypted result that, when decrypted, matches the result of operations performed on the plain text.

Let's denote \( \xi_k \) as the encryption function, \( \xi_k^{-1} \) as the decryption function, and \( f \) as a function that can be computed on plain texts. Homomorphic Encryption satisfies the property:

\[
    \xi_k(f(x)) = f(\xi_k(x))
\]

During the years, several Homomorphic Encryption schemes have been proposed, each with different properties and capabilities. The scheme used during our tests is the Brakerski-Gentry-Vaikuntanathan (BGV, 2011) scheme \cite{Brakerski2012-wj}, which is a leveled fully homomorphic encryption scheme that supports both addition and multiplication operations on encrypted data. The BGV scheme is particularly notable for its efficiency and ability to handle large integers, making it suitable for practical applications in privacy-preserving computations.
This scheme was based on the security of \textbf{(Ring) Learning With Errors} (RLWE) \cref{sec:rlwe} problem, which is a hard problem in lattice-based cryptography. The need for such a scheme arises from the increasing demand for secure computations in various fields, including cloud computing, data analysis, and machine learning. This type of new technology is designed to resist quantum computer and cryptanalysis. 

\subsubsection{Security foundation: (Ring) Learning With Errors (RLWE)} \label{sec:rlwe}

The security of lattice-based FHE schemes, especially BGV, rests on the hardness of the Ring Learning With Errors (RLWE) problem, a ring-based variant of the Learning With Errors (LWE) problem introduced by Lyubashevsky, Peikert, and Regev in 2010 \cite{Lyubashevsky2010-jo}.
. In RLWE, one works over a polynomial ring modulo both a prime $q$ and an irreducible polynomial $f(x)$:

\[
    a(x) = a_0 + a_1 x + \ldots + a_{n-1} x^{n-1}, \text{where } a_i \in \mathbb{Z}_q
\]

Samples are of the form $(a(x),\,b(x)=a(x)s(x)+e(x))$, where $e(x)$ is a small “error” polynomial. Recovering $s(x)$ given many such samples is presumed hard, based on reductions to the Shortest Vector Problem (SVP) in ideal lattices.

\subsubsection{Homomorphic Encryption Types}

\textbf{Fully Homomorphic Encryption (FHE)} allows the evaluation of arbitrary circuits of additions and multiplications over encrypted data, without decryption. However, earlier FHE schemes suffered from inefficiencies, particularly due to large growth of noise. The BGV scheme answered these challenges by avoiding Gentry’s “bootstrapping” \cite{brakerski2011leveled} step via \emph{leveled} evaluation, and controlling noise through \emph{modulus switching} and \emph{relinearization} techniques.

\textbf{Partially Homomorphic Encryption (PHE)} supports only one type of operation—either addition (e.g., Paillier) or multiplication (e.g., RSA variants). These are faster than FHE but limited in expressiveness, making them suitable for simpler tasks where only one operation type is required.

\section{Usage of HE for Matching}
Homomorphic Encryption can be used to securely match location. To archive this, we can use different techniques such as:
\begin{itemize}
    \item \textbf{Distance Calculation}: It is archived by tweaking the distance calculation algorithms (mentioned before in \cref{sec:background-distances} e.g., Euclidean distance, Cosine similarity, Manhattan Distance) to work with encrypted coordinates. The main limit comes with the constraint that the operations must be compatible with the encryption scheme used. For instance, calculating the squared Euclidean distance between two points \( (x_1, y_1) \) and \( (x_2, y_2) \) can be expressed as:
    \[
        d^2 = (x_1 - x_2)^2 + (y_1 - y_2)^2
    \]
    That can be computed homomorphically, and then decrypted before computing the square root to get the actual distance. We will further discuss this in the Testing Section \ref{sec:testing-performance}.
    \item \textbf{Encoding Coordinates}: By encoding coordinates into a grid-like structure, we can reduce the problem into finding two different points sharing the same area-encoding. This approach comes with some limitations, mainly related to the precision of the approximation and the size of the grid. Still, it allows for efficient matching of points within a specific area without revealing their exact coordinates. This is done by applying a space-filling curve, such as Z-order \cref{sec:z-order-curve}, to the coordinates.
\end{itemize}

\subsubsection{Privacy-Preserving Z-order Queries}
When implementing privacy-preserving location queries using Z-order encoding, we can employ homomorphic encryption to protect sensitive location data \cite{zhang2020privacy}. Let's consider a scenario where a user \( A \) wants to share their location with user \( B \) while maintaining privacy:

\begin{itemize}
    \item Let \( QK_A \) be the QuadKey representation of \( A \)'s location
    \item Let \( M_B \) be the bit mask specified by \( A \) for user \( B \)
    \item The service provider (SP) performs homomorphic multiplication: \( QK_A \otimes M_B \)
\end{itemize}

To ensure unambiguous results, we increment each value in \( qk_i \in QK_A \) by one before encryption, such that \( qk_i \in \{1,2,3,4\} \). In this scheme \cite{zhang2020privacy}:
\begin{itemize}
    \item A bit mask value of 1 preserves the location data
    \item A bit mask value of 0 masks the location data
\end{itemize}

After decryption on the client device, we:
\begin{enumerate}
    \item Remove any zero values
    \item Convert the elements back to \( \mathbb{Z}_4 \)
    \item Generate a masked QuadKey string that produces a bounding box with the desired level of detail
\end{enumerate}

This approach is computationally efficient as it requires only one round of homomorphic multiplication. The resulting bounding box effectively hides both precise locations and movement patterns, providing privacy even against colluding users.

For example, given a precise GPS coordinate \( (43.084451, -77.680069) \), the system can generate bounding boxes of varying sizes based on the privacy preferences (where \( d \) represents the level of detail). This ensures that location data is shared at an appropriate granularity while maintaining user privacy.

% TODO: add other type of queries

\subsection{PHE and FHE Implications}
The choice between Partially and Fully Homomorphic Encryption has significant implications for system performance, security, and functionality. As previously discussed, the two methods used for proximity checks in a location-based scenario require careful consideration of the encryption scheme.

If the requirements only involve checking whether two positions share a common area, leveraging the speed and simplicity of the Z-order approach is recommended. Conversely, if the goal is to compute a precise floating-point distance value, a homomorphically encrypted distance function (e.g., Euclidean distance) may be necessary. However, it is important to note that in this case, the final result may be affected by computational noise inherent to FHE operations.



\include{chapters/3-system-architecture}

\definecolor{dkgreen}{rgb}{0,0.6,0}
\definecolor{gray}{rgb}{0.5,0.5,0.5}
\definecolor{mauve}{rgb}{0.58,0,0.82}

\chapter{Testing}
\section{Testing Methodology}

For the testing of the protocol, we implemented a proof-of-concept in Python, using the OpenFHE library\cite{openFHE}. The testing methodology consists of the following steps. Firstly, we generate a set of random positions for the parking spots. After that, we generate a random position for the client, we compute the z-order encoding for both the client and the parking spots, and finally we computed the distances between the client position and the parking spots using the homomorphic encryption scheme.

To ensure a fair comparison, we test different scenarios with different sets of hyper parameters, such as the number of parking spots, the number of clients running simultaneously, a range of network delays, and the edge size in meters of a grid cell (used to encode the GPS coordinates). The goal is to measure the performance of the homomorphic encryption scheme in terms of encryption and decryption time, as well as the time taken to compute the distances between the client position and the parking spots.

At the same time, we also computed the distance calculation using a clear computation (trusted backend) with encrypted with asymmetric encryption (RSA) traffic. This allows us to compare the performance of the homomorphic encryption scheme against a traditional asymmetric encryption scheme.

The testing environment consists of a virtual machine instance on a server, with the following specifications:
\begin{itemize}
    \item CPU: Intel(R) Xeon(R) Silver 4110 CPU @ 2.10GHz 16 cores
    \item RAM: 64 GB
    \item OS: Debian 11
    \item Python version: 3.8
    \item OpenFHE version: 1.3.0.0.20.4
    \item numpy version: 1.24.0
\end{itemize}

\section{Performance Evaluation} \label{sec:testing-performance}

The first test that we perform is to measure the time taken to encrypt and decrypt the client position using the homomorphic encryption scheme. The results are shown in Figure \ref{fig:testing-enc}. The time taken to encrypt around 50 parkings with the same key is around 0.5 seconds, while the time taken to decrypt the client position is around 0.2 seconds. This resoults that the homomorphic encryption scheme is efficient for this use case, but it still grows linearly with the number of data points.

\begin{figure}[h]
    \centering
    \includegraphics[width=\columnwidth]{img/crypto_times.png}
    \caption{Visualization of the Encryption and Decryption time using HE}
    \label{fig:testing-enc}
\end{figure}

The test begins with the generation of the parameters for the homomorphic encryption scheme, followed by the key generation. The code snippet in Listing \ref{lst:enc-decrypt} shows the code used to encrypt and decrypt the client position.
% add the code to encrypt
\lstset{frame=tb,
  language=python,
  aboveskip=3mm,
  belowskip=3mm,
  showstringspaces=false,
  columns=flexible,
  basicstyle={\small\ttfamily},
  numbers=none,
  numberstyle=\tiny\color{gray},
  keywordstyle=\color{blue},
  commentstyle=\color{dkgreen},
  stringstyle=\color{magenta},
  breaklines=true,
  breakatwhitespace=true,
  tabsize=3
}

\newpage

\begin{lstlisting}[language=python, caption={Encryption and Decryption of the client position}, label={lst:enc-decrypt}]
parameters = CCParamsBGVRNS()
parameters.SetPlaintextModulus(65537)
parameters.SetMultiplicativeDepth(2)

crypto_context = GenCryptoContext(parameters)
crypto_context.Enable(PKESchemeFeature.PKE)
crypto_context.Enable(PKESchemeFeature.KEYSWITCH)
crypto_context.Enable(PKESchemeFeature.LEVELEDSHE)

keypair = crypto_context.KeyGen()
crypto_context.EvalMultKeyGen(keypair.secretKey)
\end{lstlisting}


The next step is to compute the distances between the client position and the parking spots. The code snippet in Listing \ref{lst:compute-distances} shows the code used to compute the distances using the homomorphic encryption scheme.

\begin{lstlisting}[caption={Computing distances using Homomorphic Encryption}, label={lst:compute-distances}]
# Encrypt the client position
query_plaintext = crypto_context.MakePackedPlaintext([query_encoded])
query_ciphertext = crypto_context.Encrypt(keypair.publicKey, query_plaintext)

# Process each spot
spots = parking_system.get_spots()

for spot_id, spot in spots.items():
    # Create plaintext for the spot's encoded position
    spot_plaintext = crypto_context.MakePackedPlaintext([spot['encoded_pos']])
    
    # Homomorphic subtraction to check if positions match
    diff_ciphertext = crypto_context.EvalSub(query_ciphertext, spot_plaintext)
                
    diff_plaintext = crypto_context.Decrypt(keypair.secretKey, diff_ciphertext)

\end{lstlisting}

This two sections from the original codebase showcase how the encryption distance calculation works. The full code contains additional logic for handling metrics and logging.

In order to have a fair comparison with other common encryption standards, I made the test run in parallel by multi-processing on the same machine. This refiniment made the test more realistic, as in a real-world scenario, multiple clients would be sending requests to the server at the same time with multiple parking spots (subscribers).

\begin{figure}[h]
    \centering
    \includegraphics[width=12.5cm,height=9cm]{img/total_time_comparison.png}
    \caption{Comparison between RSA and HE for encryption and decryption}
    \label{fig:he-vs-rsa}
\end{figure}

The overall time performance of the protocol is shown in \cref{fig:he-vs-rsa} confirms that the HE standard has some overhead compared to the RSA standard, but it is still acceptable for a real-world scenario. The main difference between the two encryption standards is that, with the symmetric encryption, the server, after decrypting the client position, is much faster in computing the distances, as it can be achieved in a couple of instructions.

Although the homomorphic encryption needs to perform a simple subtraction, it is still slower, as values need to be packed to match HE standards \ref{lst:compute-distances}.

It also worth to mention the way the distances are computed in our protocol. Firstly, we need to encode the GPS coordinates into a grid-like structure. This operation is essential to ensure that we are considering small areas instead of points \cref{lst:distance-computation}. Then we apply a unique encoding to the grid coordinates, transforming them from a two-dimensional grid into a one-dimensional z-order curve. By having a single integer representation we can perform easily the distance.

\begin{lstlisting}[caption={Distance computation using z-order encoding}, label={lst:distance-computation}]
def calculate_grid_sizes_for_radius(radius_meters, max_lat, min_lat, max_lon, min_lon):
    radius_km = radius_meters / 1000.0
    lat_span = max_lat - min_lat
    lon_span = max_lon - min_lon
    
    # Latitude: fixed ~111.32 km/degree
    lat_cells = int(lat_span / (radius_km / 111.32))
    
    # Longitude: depends on latitude
    mean_lat = math.radians((max_lat + min_lat) / 2)
    lon_cells = int(lon_span / (radius_km / (111.32 * math.cos(mean_lat))))
    
    return lat_cells, lon_cells  # Return separate sizes

def normalize_gps(lat, lon, lat_cells=None, lon_cells=None, edge_meters=500, max_lat=44.499194, min_lat=44.492307, max_lon=11.363250, min_lon=11.325319):

    if lat_cells is None or lon_cells is None:
        lat_cells, lon_cells = calculate_grid_sizes_for_radius(edge_meters, max_lat, min_lat, max_lon, min_lon)
    
    grid_x = int((lat - min_lat) / (max_lat - min_lat) * lat_cells)
    grid_y = int((lon - min_lon) / (max_lon - min_lon) * lon_cells)
    
    return grid_x, grid_y
\end{lstlisting}

To compute the z-order encoding, we use a straightforward approach that interleaves the bits of the x and y coordinates. The code snippet in Listing \ref{lst:z-order-encoding} shows how we compute the z-order encoding for the grid coordinates.

\begin{lstlisting}[caption={Z-order encoding for grid coordinates}, label={lst:z-order-encoding}]
def interleave_bits(x, y):
    """
    Interleave the bits of x and y to create a Morton code.
    x and y should be non-negative integers, each limited to 16 bits.
    """
    # Ensure inputs are within 16-bit range
    x = min(x, 0xFFFF)
    y = min(y, 0xFFFF)
    
    # Convert to binary and pad with zeros to 16 bits
    x_bin = format(x, '016b')
    y_bin = format(y, '016b')
    
    # Interleave the bits
    result = ''
    for i in range(16):
        result += x_bin[i] + y_bin[i]
    
    return int(result, 2)
\end{lstlisting}


There are also other approaches to compute distances between two encrypted points\cite{ibarrond2022hedistancetricks}, as mentioned for the methods in \cref{sec:background-distances}:
\begin{itemize}
    \item \textbf{Cosine similarity:} This can be resolved by normalizing the vectors, i.e., $\mathbf{x}' = \mathbf{x} / \|\mathbf{x}\|$ and $\mathbf{y}' = \mathbf{y} / \|\mathbf{y}\|$, encrypting $\mathbf{x}'$ and $\mathbf{y}'$, and then performing a simple scalar product: $\sum_i x'_i y'_i$.
    
    \item \textbf{Euclidean Distance:} would require a square root operation.\\
    Instead we use the \textbf{Squared Euclidean Distance} instead:
    $
    \mathrm{SED}(\mathbf{x}, \mathbf{y}) = \sum_i (x_i - y_i)^2
    $
    
    \item \textbf{Manhattan Distance:} in this case it is also require computing the absolute value.\\
    If you encrypt only binary values (i.e., $\hat{\mathbf{x}}, \hat{\mathbf{y}}$ such that $\hat{x}_i, \hat{y}_i \in \{0,1\}$ for all $i$), you can reformulate:
    $
    \mathrm{MD}(\hat{\mathbf{x}}, \hat{\mathbf{y}}) = \sum_i (\hat{x}_i - \hat{y}_i)^2 = \mathrm{HD}(\hat{\mathbf{x}}, \hat{\mathbf{y}})
    $
    which is the \textbf{Hamming Distance} . For non-binary vectors, you can at least compute the \textbf{Squared Manhattan Distance}
    $
    \mathrm{SMD}(\mathbf{x}, \mathbf{y}) = \sum_i (x_i - y_i)^2
    $
    but this is missing a square root to recover the standard Manhattan distance.
    
\end{itemize}

Rounding off, all of these methods have the problems of relying on calculations on floating point numbers, which is not ideal for homomorphic encryption. Another way could be to normalize the coordinates to integers, with a fixed order of magnitude, but it is not worth the effort, as the z-order encoding is already a good solution for this problem.



\chapter{Conclusion}

\section{Summary}

\subsection{Results}

From the testing phase, we can conclude that the protocol is feasible and functional. The performance of the homomorphic encryption scheme is acceptable for a real-world scenario, although it is still computationally more expensive than a traditional RSA solution. However, it can be applied in a zero-trust scenario, where the server cannot be trusted to handle sensitive data. The introduction of PRE (Proxy Re-Encryption) allows us to offload the computation to a trusted third party, which can be used to compute the distances between the client position and the parking spots without revealing the client position to the server. Moreover, this resolved the previous issue of the server possible leak of client position, under the threat model~\cref{subsec:threatmodel}.

\subsection{Protocol Scalability}

One more factor to consider is the scalability of the protocol. The MCs and LDs can be scaled horizontally, meaning that we can add more servers to handle more clients and parking spots. The same applies to the server and workers; for instance, we can have multiple servers handling different areas of the metropolis, each with its own set of parking spots. However, the CA (Certificate Authority) is not. Moreover, the CA could have multiple instances, but they would need to be synchronized in order to issue certificates for the same client.

\subsection{Protocol Limitations}

In conclusion, the protocol presented in this thesis is a feasible solution for the problem of finding nearby parking spots in a zero-trust scenario. This was possible thanks to the correct usage of the right network protocols, such as the publish/subscribe and request/response protocols. During my work, I concluded that even though the publish/subscribe protocol is preferred for this kind of applications, it does not match the requirements of a zero-trust scenario. Moreover, working with homomorphic encryption adds the requirement that the server is not allowed to know the plain text of the clients. This is a crucial point, as it does not allow us to infer informations from the client data. So, the request/response protocol works better in this sense: the client request anonymously a service, the server responds applying a list of operations to the encrypted data, and then the client can decrypt the result. In the opposite case, the server would need to know \emph{when} to respond to the client, which is most of the time a consequence of a side-effect to the request. Let's consider the case the LA-MQTT protocol \cref{sec:la-mqtt}: the client subscribes to a topic, and the server publishes the result based on the client location. In this case, is obvious that the server needs to know the client position in order to publish the result. Indeed, if this was the case, we would not be able to apply the homomorphic encryption scheme.

\section{Future Works}

During the testing phase, we identified several areas for improvement for future work. One of them is the possible integration of IoT specific libraries, such as the \emph{SEAL Embedded} \cite{sealembedded} library, which is designed for resource-constrained devices. This would allow us to run the protocol on edge devices, such as sensors or micro-controllers, without the need for a powerful server. The obvious constraint is that the whole system should be adapted to work with the Microsoft SEAL\cite{sealcrypto} Library also on the server side. 

Another huge area of possible optimization is the adjacency calculation using z-order encoding and FHE. In the paper \cite{zhang2020privacy}, they proposed a standard for computing useful queries on homomorphic encrypted position-related data. In their work, they were considering a social media application, that was collecting user positions to show nearby friends. Their work not only showcased that you can apply z-order encoding to encode positions, but comes really handy when using preference levels for positions sharing. Let's consider the following example: a user wants to share his position with a friend, he does not trust him enough to share his exact location. Instead, he can apply a preference bit-mask in order to share only a certain order level of the z-order encoding. The paper also shows that is possible to compute the exact distance between two positions: the main idea is to create the smallest encoded box that contains both positions even in FHE. This concept could be applied to our protocol, allowing us not only to find parkings sharing our same area, but also to compute the distance between a particular parking.

Our encoding standard is based on the transformation of GPS coordinates into a z-order curve, thus an approximate representation of an area. When applying this encoding, we need to consider the order of magnitude of the error introduced by the encoding. That doesn't always mean that the error is negligible, but at the same time could also work in our favor, as it allows to obfuscate an exact position.



\renewcommand{\bibsection}{}
\chapter*{References}
\bibliography{refs}
\newpage

\renewcommand{\appendixtocname}{Appendices}
% \csname @openrightfalse\endcsname
\pagenumbering{gobble}

\newpage~\newpage
\chapter*{Ringraziamenti}
Grazie a tutti
\end{document}
