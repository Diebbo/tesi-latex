\documentclass[12pt,a4paper,twoside]{book}


\usepackage[utf8]{inputenc}
\usepackage[a4paper,inner=3.5cm,outer=2.5cm]{geometry}

\usepackage[titletoc,title,toc,page]{appendix}
\usepackage{verbatim}
\usepackage{placeins}
\usepackage{listings}
\usepackage{hyperref}
\usepackage[english]{babel}
\usepackage{tikz}
\usepackage{parskip}

\usepackage{graphicx}
\usepackage{blindtext}
\usepackage{chngcntr}
\counterwithin{table}{chapter}

\usepackage{newlfont}
\usepackage{fancyhdr}
\usepackage{indentfirst}
\usepackage[utf8]{inputenc}
\usepackage{float}
\usepackage{hyperref}
\usepackage[capitalize,noabbrev]{cleveref}
\usepackage{soul}
\usepackage[font=footnotesize,labelfont=bf]{caption}

\usepackage{multirow}
\usepackage{hyphenat}
\hyphenation{mate-mati-ca recu-perare}

\usepackage{lscape} 

\usepackage{natbib}
\bibliographystyle{alpha}
\setcitestyle{super,open={[},close={]}}

\newcommand{\rom}[1]{\uppercase\expandafter{\romannumeral #1\relax}}

\usepackage{pdfpages}

\begin{document}
% Per spostare i vari elementi più su o più giù gioca con i valori di vspace che ci sono tra uno e l'altro
\pagestyle{empty}
\newgeometry{
    left=20mm,
    right=20mm,
    top=20mm,
    bottom=20mm
}

\begin{titlepage}

\begin{center}

% marchio di ateneo
\includegraphics[width=6.5cm,height=4.7cm]{img/marchio-di-ateneo.png}

\vspace{10mm}

% \large is 12pt
{\large{\bf{Dipartimento di Scienze e Ingegneria informatica}}} 

\vspace{5mm}

% \Large is 14.4pt
{\Large{\bf{Corso di Laurea in Informatica}}}

\vspace{15mm}

{\Huge{\bf Homomorphic Encryption Applications }}\\
\vspace{3mm}
{\Huge{\bf on the LA-MQTT Protocol}}\\
\vspace{3mm}

\end{center}

\vspace{10mm}

\begin{minipage}[t]{0.40\textwidth}
{\Large{\bf Relatore: \\ Chiar.mo Prof.\\ Federico Montori}}

\vspace{3mm}

{\Large{\bf Correlatore: \\ Chiar.mo Prof.\\ Saverio Giallorenzo}}
\end{minipage}
\hfill
\begin{minipage}[t]{0.40\textwidth}\raggedleft
{\Large{\bf Presentata da: \\ Diego Barbieri}}
\end{minipage}

\vspace{30mm}

\rule[0.5cm]{15.8cm}{0.6mm}

\begin{center}
{\large{\bf Sessione mese anno \\}}
{\large{\bf Anno Accademico 2024/2025\\}}
\end{center}

\end{titlepage}

\restoregeometry
\newpage
\begin{center}
    (DA FARE ALLA FINE)\\
    5 parole chiave per caratterizzare il contenuto della dissertazione:\\ (se non ti piacciono così sparse puoi anche semplicamente scriverle su una riga sola)
\end{center}

% https://tex.stackexchange.com/questions/26538/words-scattered-randomly-in-on-coverpage
\begin{tikzpicture}[overlay,remember picture,shift=(current page.center)]
\pgfmathsetseed{3}


\foreach [count=\count] \word in {Parola 1, parola 2, parola 3, parola 4, parola 5} {
\node [
    xshift={(mod(\count,3)-1)*(\paperwidth/4)},
    yshift={(mod(\count,7)-3)*(\paperwidth/6)},
    xshift=rand*4cm,
    yshift=rand*2cm,
    % rotate=rand*35,
    % opacity=rnd*0.5+0.125,
    font=\large] {\word};
}
\end{tikzpicture}
\newpage

\topmargin=6.5cm
\begin{flushright}
\emph{
\LARGE{La dedica}\\\vspace{2mm}
\LARGE{anche quella se vuoi}\\\vspace{3mm} 
\LARGE{su più righe} 
}
\end{flushright}
\newpage~\newpage
\pagenumbering{gobble}
\chapter*{Abstract}
Abstract qui (ti consiglio di farlo alla fine)

\topmargin=-1cm
\tableofcontents
\thispagestyle{empty}
\listoftables
\thispagestyle{empty}
\listoffigures
\thispagestyle{empty}
\newpage~\newpage


\pagenumbering{arabic}
\setcounter{chapter}{-1}
\raggedbottom
\chapter{INTRODUZIONE} \label{chap:intro}
\pagestyle{plain}
\setcounter{page}{1}




\chapter{Background}
\section{Protocoll Request-Response}
\section{Protocoll Publish-Subscribe}
\subsection{MQTT}
\subsection{LA-MQTT}
\section{Universal location referencing}
\subsection{Cantor Pairing}
\subsection{MGRS}
\section{Space Filling Curves}
\section{Privacy Preserving}
\subsection{Homomorphic Encryption}
\subsubsection{Fully Homomorphic Encryption}
\subsubsection{Partially Homomorphic Encryption}
\section{Usage of HE for matching}
\subsection{Distance Calculation}
\subsection{Parkings Availablility}
\subsection{PHE and FHE implications}

\chapter{System Architecture}
\section{System Overview}
\section{Actors}
\subsection{LDS}
\subsection{MC}
\subsection{Broker}
\subsection{Worker}
\section{Client}
\section{Server}

\chapter{Protocol Design}
\section{Protocol Overview}
\subsection{Position Publish}
\subsection{Area Subscription}
\subsection{Geofence Publish}

\chapter{Testing}

\chapter{Performance Evaluation}

\chapter{Progetto}
\section{Glossario}
\begin{itemize}
    \item LDS: Sensori
    \item MC: Clienti
\end{itemize}

\section{Idea Progettuale}
L'idea centrale consiste nell'applicare i principi della \emph{homomorphic encryption} al protocollo \emph{LA-MQTT}. Quando un client invia un aggiornamento di posizione, questa viene trasmessa in forma cifrata mediante Fully Homomorphic Encryption (FHE). Il server esegue un'analisi sulle posizioni disponibili, appartenenti alla stessa area geografica, e dopo aver effettuato un controllo con la chiave di verifica, restituisce un blob cifrato contenente le distanze calcolate.

\subsection{Funzionalità Principali}
\begin{itemize}
    \item \textbf{Position Publish}: Il client aggiorna la propria posizione per ricevere informazioni dai publisher relativamente ai sensori presenti nell'area di interesse.
    \begin{itemize}
        \item $Enc(GPS\_DATA)$
        \item $P_{i}$: Posizione
    \end{itemize}
    
    \item \textbf{Area Subscription}: Un client si sottoscrive a un'area di interesse. Per ogni aggiornamento di posizione viene effettuato un controllo su tutte le posizioni riportate fino a quel momento appartenenti all'area.
    \begin{itemize}
        \item $C(i, t_{s})$
    \end{itemize}
    
    \item \textbf{Geofence Publish}: Permette ai client LDS di pubblicare un blob contenente:
    \begin{itemize}
        \item Dati geofence (formato GEOJSON RFC 7946)
        \item $g_s$: geofence target
        \item $c_{s}$: contenuto pubblicizzato
        \item $t_{s}$: topic
        \item $e_{s}$: evento di mobilità
    \end{itemize}
\end{itemize}

\section{Attori del Sistema}
\begin{itemize}
    \item \textbf{Broker/Proxy}: Gestisce le chiavi di tutti i client che hanno effettuato almeno una sottoscrizione
    \item \textbf{Publisher}: Pubblicano aggiornamenti sulle posizioni disponibili
    \item \textbf{Subscriber}: Sottoscrivono aree geografiche e ricevono aggiornamenti sui parcheggi disponibili
\end{itemize}

\section{Vantaggi}
\begin{itemize}
    \item Dati sempre cifrati
    \item In caso di \emph{re-encryption}, solo il proxy conosce le posizioni decifrate dei sensori
    \begin{itemize}
        \item Un proxy malevolo non compromette la posizione dell'utente
        \item Possibile individuare proxy fittizi mediante falsi aggiornamenti di posizione
    \end{itemize}
\end{itemize}

\section{Svantaggi}
\begin{itemize}
    \item Latenze significative (worker)
    \item Attenzione necessaria alla triangolazione delle posizioni
\end{itemize}

\section{Stack Tecnologico}
\begin{itemize}
    \item \textbf{Soluzione ad alte prestazioni}:
    \begin{itemize}
        \item C++ SEAL (Microsoft)
        \item Rust TFHE-rs (ZAMA)
    \end{itemize}
    \item \textbf{Proof of Concept}:
    \begin{itemize}
        \item Python Concrete (ZAMA)
    \end{itemize}
\end{itemize}

\section{Interrogativi Aperti}
\begin{itemize}
    \item \textbf{Prestazioni}: È accettabile un delay di 5-10 secondi per l'analisi dei dati?
    \item \textbf{Re-encryption}: È una soluzione eccessiva?
    \begin{itemize}
        \item Il server salva le posizioni cifrate nel database
        \item Possibile alternativa: Private Set Intersection (PSI)
    \end{itemize}
    \item \textbf{Architettura}: Request-Response vs MQTT
\end{itemize}

\section{Sviluppi Futuri}
\begin{itemize}
    \item Integrazione della libreria \emph{SEAL Embedded} per i sensori
    \item Definizione della funzione per il calcolo delle distanze
    \item Analisi degli ordini di grandezza degli errori
    \item Selezione delle posizioni rilevanti:
    \begin{itemize}
        \item Area poligonale (due punti)
        \item Area discreta (MGRS)
    \end{itemize}
    \item Deduplicazione di vettori numerici
    \item Ottimizzazione delle richieste per evitare attacchi DoS
    \begin{itemize}
        \item Dimostrazione delle vulnerabilità nell'implementazione precedente
        \item Proposta di patch risolutive
    \end{itemize}
\end{itemize}


\section{riferimenti}
Come detto prima le label servono per riferirsi ad altre parti del testo citate precendentemente.\\
Ti consiglio di metterle sempre almeno a figure. immagini e capitoli.

Per riferirti a qualcosa basta fare ref seguito dal nome della label, ad esempio ``vedi capitolo \ref{chap:intro}''.\\In questo modo dal pdf cliccando sulla reference, ti porta direttamente al punto giusto.
Altri pacchetti come \texttt{fancyref} e \texttt{cleveref} (consigliato) possono aiutare nell'automatizzare la creazione delle refrence. Usando ad esempio \texttt{\cref{chap:intro}} viene generata la dicitura corrispondete all'elemento a cui si fa riferimento, seguita dalla numerazione. Eccone un esempio: \cref{chap:intro}.
\section{citazioni}
Per citare si usa cite seguito dal nome dell'articolo nel file.bib, ad esempio ``come visto nell'articolo di tizio\cite{greenwade93}''.

Se non ti piace lo stile di citazione puoi modificarlo sopra dove scrivo usepackage natbib, ma quello impostato attualmente dovrebbe andare bene.



\renewcommand{\bibsection}{}
\chapter*{Riferimenti bibliografici}
\bibliography{refs}
\newpage

\renewcommand{\appendixtocname}{Appendici}
\renewcommand{\appendixpagename}{Appendici}
% \csname @openrightfalse\endcsname
\pagenumbering{gobble}

\newpage~\newpage
\chapter*{Ringraziamenti}
Grazie a tutti
\end{document}
